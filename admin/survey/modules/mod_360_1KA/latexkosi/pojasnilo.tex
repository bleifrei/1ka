\chapter{Metodologija}
\subsection{Vprašalnik kompetenc po metodi 360°} 
Kompetence so skupek povezav med znanjem, veščinami in vedenji, ki vplivajo na delo posameznika in korelirajo z uspešnostjo na delovnem mestu. So skupek povezav, ki so merljive z dobro sprejetimi standardi in ki jih lahko izboljšujemo z izobraževanjem in razvojem zaposlenih. Z vprašalnikom kompetenc merimo sledeče dimenzije: 


\begin{itemize}
\item Komuniciranje
\item Sposobnosti odločanja in presoje
\item Vodenje in ravnanje z ljudmi
\item Vodenje projektov
\item Medosebne veščine
\end{itemize} 
\bigskip
 \textbf{Metoda 360 stopinj} je orodje, s katerim merimo prisotnost oziroma izraženost kompetenc pri posamezniku. Pomaga nam identificirati posameznikove prednosti in njegove šibke točke ter odkriti potenciale za razvoj ter možnosti za izboljšave. Ta metoda je kompleksna in zanesljiva, saj pri njej ocenjujemo kompetence posameznika z več vidikov, z vidika:
\begin{enumerate}
  \item nadrejenih,
  \item sodelavcev na enakem nivoju,
  \item podrejenih,
  \item ter z vidika ocenjevanca (samoocena).
\end{enumerate}
Na ta način dobimo celostno sliko, ki je lahko v pomoč pri določanju nadaljnjih razvojnih ciljev za posameznika na področjih, kjer se je pokazalo, da je še prostor za izboljšave. 
